% LaTeX resume using res.cls
\documentclass[geomargin]{res}
\usepackage{amssymb,amsmath}
\usepackage{hyperref}
%\usepackage{helvetica} % uses helvetica postscript font (download helvetica.sty)
%\usepackage{newcent}
\usepackage{multicol}
\usepackage{ccfonts}
%\setlength{\textwidth}{6.5in} % set width of text portion
%\setlength{\oddsidemargin}{-0.25in}
\usepackage[left=.3in,top=0.6in,right=2in,bottom=.02in]{geometry}
\renewcommand{\labelitemi}{\tiny$\blacksquare$}
\usepackage[normalem]{ulem}

% set up href options
\hypersetup{
  pdfnewwindow=true,
  colorlinks=true,
  urlcolor=[rgb]{0.5451, 0, 0}
}

\begin{document}
%%%%%%%%%%%%%%%%%%%%%%%%%%%%%%%%%%%%%%%%%%%%%%%%%%%%%%%%%%%%%%%%%%%%%%%%%%%%%%
%%%%%%                               Heading                            %%%%%%
%%%%%%%%%%%%%%%%%%%%%%%%%%%%%%%%%%%%%%%%%%%%%%%%%%%%%%%%%%%%%%%%%%%%%%%%%%%%%%
\moveleft.5\hoffset\centerline{\huge \bf GUANG YANG}            % Center the name over the entire width of resume:
\moveleft\hoffset\vbox{\hrule width 7.3in height 1pt}         % Draw a horizontal line the whole width of resume:
\address{\url{http://www.github.com/guang} \\San Mateo, CA} % address begins here
\address{ \hfill \url{https://www.linkedin.com/in/gyang8/} \\ \hfill garry.yangguang@gmail.com}
\begin{resume}

%%%%%%%%%%%%%%%%%%%%%%%%%%%%%%%%%%%%%%%%%%%%%%%%%%%%%%%%%%%%%%%%%%%%%%%%%%%%%%
%%%%%%                          Experiences                             %%%%%%
%%%%%%%%%%%%%%%%%%%%%%%%%%%%%%%%%%%%%%%%%%%%%%%%%%%%%%%%%%%%%%%%%%%%%%%%%%%%%%
\section{EXPERIENCE}

{\bf INSIGHT DATA SCIENCE,} Palo Alto, CA \\
{\em Artificial Intelligence Fellow} \hfill
Jun 2018 -- Present \\                                          \vspace{-4mm}
% \begin{itemize}                                         \itemsep1pt %\small
%   \item 
% 
% \end{itemize}
\vspace{-1mm}

{\bf KOMODO HEALTH,} San Francisco, CA \\
{\em Data Engineer} \hfill
Mar 2015 -- May 2017 \\
As employee \#3, I have had the opportunity to add value in different parts of the engineering stack, from scratch: \\                                          \vspace{-4mm}

\begin{itemize}                                         \itemsep1pt %\small
  \item Designed and implemented distributed (Spark, S3) and automated (Airflow) data pipelines that process multiple external data sources to deliver mission-critical insights to end-users

  \item Led implementation of internal services (containerized Flask APIs deployed on Kubernetes) that improve transparency (domain-knowledge driven data metrics tracking) and automation (config and metadata tracking)

  \item Built and maintained the data infrastructure (AWS, Spark, Airflow, internal services) that support the data team (13+ developers), leveraging container technologies (Kubernetes, Docker)

  \item Worked with architect and product managers to define requirements and roadmap for scalability of data deployment, data ingestion and internal tooling.

\end{itemize}
\vspace{-1mm}

{\bf INSIGHT DATA SCIENCE,} Palo Alto, CA \\
{\em Data Engineering Fellow} \hfill
Jan 2015 -- Mar 2015 \\                                          \vspace{-4mm}
\begin{itemize}                                         \itemsep1pt %\small
  \item \href{https://github.com/guang/stargazer}
    {Built a data pipeline for map analytics in Starcraft II\(^\copyright\)
     using replay files.}

  \item Gained experience in large scale distributed data architectures through training,
    mentorship and projects using Kafka, Spark, Cassandra, and Hadoop ecosystem tools.
\end{itemize}
\vspace{-1mm}

{\bf BERKELEY COMPUTATIONAL OPTIMIZATION LAB,} Berkeley, CA \\
{\em Graduate Student Researcher} \hfill
Aug 2013 -- May 2014 \\                                          \vspace{-4mm}
\begin{itemize}                                         \itemsep1pt %\small
        \item    Formulated and implemented an algorithm to analyze exact reachability for skew-line needle planning in automated brachytherapy
        \item    \href{http://ieeexplore.ieee.org/abstract/document/6899376/}{Published results in an article for the IEEE CASE 2014 Conference}
\end{itemize}
\vspace{-1mm}


% {\bf NATIONAL INSTITUTE FOR MATHEMATICAL AND BIOLOGICAL SYNTHESIS,} Knoxville, TN \\
% {\em NSF, Research Experience for Undergraduates}    \hfill
% Jun -- Aug 2010 \\                                          \vspace{-4mm}
% \begin{itemize}                                         \itemsep1pt 
%         \item Built a model in R to simulate the dynamics of Johne's Disease in a U.S. dairy herd. Performed cost analysis comparing existing control strategies and a newly developed testing method
%         \item \href{http://www.worldscientific.com/doi/abs/10.1142/S021833901340010X}{Published results in a paper for Journal of Biological Systems}
% \end{itemize}
% \vspace{-1mm}


%%%%%%%%%%%%%%%%%%%%%%%%%%%%%%%%%%%%%%%%%%%%%%%%%%%%%%%%%%%%%%%%%%%%%%%%%%%%%%
%%%%%%%                             Awards                              %%%%%%
%%%%%%%%%%%%%%%%%%%%%%%%%%%%%%%%%%%%%%%%%%%%%%%%%%%%%%%%%%%%%%%%%%%%%%%%%%%%%%
%\section{AWARDS}
%\vspace{3mm}
%\begin{itemize}
%\item
%UC Berkeley Graduate Fellowship 2013
%\item
%CAAM-Chevron Undergraduate Research Prize 2013
%\item
%Meritorious winner, COMAP Mathematical Contest in Modeling 2011
%\item
%1st Place, Rice ASME Engineering Design Competitions 2011, 2012
%\item
%NSF, Research Experience for Undergraduates 
%\begin{itemize}
%\vspace{-1.9mm}
%\item 
%Claremont Colleges, Claremont, CA (summer 2011)
%\vspace{-1.5mm}
%\item
%National Institute for Mathematical and Biological Synthesis, Knoxville, TN (summer 2010)
%\end{itemize}
%\end{itemize}
%
%\vspace{-4mm}


%%%%%%%%%%%%%%%%%%%%%%%%%%%%%%%%%%%%%%%%%%%%%%%%%%%%%%%%%%%%%%%%%%%%%%%%%%%%%%
%%%%%%                              Technical                           %%%%%%
%%%%%%%%%%%%%%%%%%%%%%%%%%%%%%%%%%%%%%%%%%%%%%%%%%%%%%%%%%%%%%%%%%%%%%%%%%%%%%
\section{TECHNICAL}

% %%%%% One Column Format %%%%%
\begin{itemize}
  \item \textit{Languages}: Python, SQL, shell
  \item \textit{Frameworks}: Spark, Airflow, Postgres, Flask, etcd
  \item \textit{Infrastructure}: AWS (architecture, networking), Kubernetes, Docker
\end{itemize}


%%%%%%%%%%%%%%%%%%%%%%%%%%%%%%%%%%%%%%%%%%%%%%%%%%%%%%%%%%%%%%%%%%%%%%%%%%%%%%
%%%%%%%                             Education                           %%%%%%
%%%%%%%%%%%%%%%%%%%%%%%%%%%%%%%%%%%%%%%%%%%%%%%%%%%%%%%%%%%%%%%%%%%%%%%%%%%%%%
\section{EDUCATION}
{\bf UNIVERSITY OF CALIFORNIA, BERKELEY,} Berkeley, CA \\
{\em M.S. in Industrial Engineering \& Operations Research }
\hfill Dec 2014\\                                     \vspace{-4mm}
\begin{itemize}
  \item Awarded UC Berkeley Graduate Fellowship
\end{itemize}
\vspace{-1mm}

{\bf RICE UNIVERSITY,} Houston, TX \\
{\em B.A. in Computational and Applied Mathematics (CAAM)}
\hfill May 2013\\                                     \vspace{-4mm}
\begin{itemize}                                         \itemsep1pt
  \item Awarded CAAM-Chevron Undergraduate Research Prize
\end{itemize}
\vspace{-1mm}


%%%%%%%%%%%%%%%%%%%%%%%%%%%%%%%%%%%%%%%%%%%%%%%%%%%%%%%%%%%%%%%%%%%%%%%%%%%%%%
%%%%%%                          Publications                            %%%%%%
%%%%%%%%%%%%%%%%%%%%%%%%%%%%%%%%%%%%%%%%%%%%%%%%%%%%%%%%%%%%%%%%%%%%%%%%%%%%%%
% \section{PUBLICATION}
% \begin{itemize}                                                 \itemsep1pt
%         \item   A.\ Garg, T.\ Siauw, \textbf{G.\ Yang}, S.\ Patil, A.\ Cunha, I.\ Hsu, J.\ Pouliot, A.\ Atamt\"urk, K.\ Goldberg. \textit{Exact Reachability Analysis for Planning Skew-Line Needle Arrangements for Automated Brachytherapy}. IEEE International Conference on Automation Science and Engineering, 2014.
%         \item  T.\ Massaro, S.\ Lenhart, M.\ Spence, C.\ Drakes, \textbf{G.\ Yang}, F.\ Agusto, R.\ Johnson, B.\ Whitlock, A.\ Wadhwa, S.\ Eda. ``\emph{Modeling for Cost Analysis of Johne's Disease Control Based on EVELISA Testing}''. Journal of Biological Systems 21, no. 04, 2013.
% \end{itemize}
% \vspace{-2mm}

\end{resume}
\end{document}
